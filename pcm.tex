\chapter{Přenos a formát multimédií}

\subsection{Přenos hlasu}

Před přenosem signálu po internetu nejdříve potřebujeme data zaznamenat a
digitalizovat. V případě audio signálu jako záznamové zařízení obvykle
slouží snímací zařízení, například mikrofon, připojené k převodníku
analogové úrovně napětí na diskrétní čísla. Pomocí analogově--digitálního
převodníku získáváme sekvenci čísel o předem nastavené vzorkovací frekvenci.
Takové kódování analogového signálu nazýváme pulzně kódová modulace, 
anglicky pulse--code modulation (PCM). Digitální kódování umožňuje ideální
rekonstrukci zaznamenaného signálu po celé přenosové cestě, takže nedochází
k poškození signálu při přenosu. Jistou nevýhodou je omezení zaznamenané 
frekvence na nejvýše polovinu frekvence vzorkovací, pokud nemá být výsledný
signál poškozen.

Nevýhodou PCM je vysoký nutný tok vzhledem k množství přenesené informace\cite{digital_speech}, oproti původnímu analogovému signálu potřebujeme vyšší přenosovou kapacitu. Proto volíme jiné způsoby kódování k ukládání a přenosu řeči.
Kódování dělíme na ztrátové, při jehož použití ztrácíme část přenášené 
informace, a bezeztrátové, kde použijeme výhodnější způsob uložení a odstraníme
nadbytečnou redundanci přenášené informace. Při záznamu hlasu obvykle 
nevyžadujeme bezchybný přenos beze ztráty informace, ale raději omezíme 
množství přenášené informace, abychom mohli komunikovat i po pomalejší lince,
nebo abychom po ní mohli přenést více toků v přijatelné kvalitě.
Protože ztrátová komprese je zaměřená na vynechání nebo redukci informace, 
kterou lidské
ucho nevnímá, nebo pro pochopení řeči není zásadní, dosahuje ztrátová komprese
výrazně lepší účinnosti, než bezeztrátová. 

Ztrátová komprese hlasu má dva základní přístupy, jakými lze přenášet hlas.
\begin{itemize}
\item{Kódování signálu tak, aby zakódovaný signál měl co nejmenší
odchylku od původního signálu byla co nejmenší. Zakódovaný signál popisuje
tvar vlnění signálu. Jeho výhodou je schopnost popsat kvalitně i obecné zvuky,
například hudbu. Nevýhodou je vyšší bitový tok, tedy nižší účinnost komprese.}
\item{Parametrické kódování signálu, popisující parametry a tvar řečového 
ústrojí člověka, pokud takový zvuk chce vydat. Nazýváme též vokodéry. 
Výhodou je nižší objem 
přenášených informací, nevýhodou obvykle vyšší výpočetní náročnost a nižší 
kvalita. }
\end{itemize}

Protože oba přístupy mají své přednosti, pro vyšší kvalitu přenosu hlasu 
zpravidla oba přístupy kompinujeme v hybridních kodecích. Hybridní kodek
popisuje některé části řeči parametricky, část obtížně popsatelnou několika
parametry zakóduje popisem vlnění. V současnosti používané kodeky jsou 
zpravidla hybridní, pokud není potřeba přenášet i neřečové signály,
nebo zachovat kompatibilitu se sítí nepodporující jiné kódování. Například 
pro přenos komunikace faxu je nutné použít kodek určený nejen pro přenos 
lidského hlasu.

