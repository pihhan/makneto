
\chapter{Návrh aplikace}

Úspěšný přenos multimédií v reálném internetu vyžaduje velké množství 
podpůrných systémů, které není vhodné implementovat vlastními silami, pokud
existují vhodné, již existující stavební prvky. V grafickém programu 
je použita jednak knihovna pro tvorbu rozhraní, v této aplikaci jde 
o~multiplatformní toolkit Qt. Další částí je podpora videa a audia, s podporou
záznamu živých zařízení, pro tento projekt jsem zvolil již zmíněný GStreamer.
Při implementaci jsem řešil propojení dvou rozdílných přístupů, protože 
GStreamer využívá knihovny GLib, psaná v jazyce C. GLib je komplexní knihovna,
která poskytuje jednak řadu rutin pro běžnou algoritmizaci, které standardní 
knihovna nenabízí. Příkladem mohou být obecné seznamy objektů, řetězců, nebo
implementované rutiny pro práci se stromem. 

Poměrně mocné je implementace podpory objektového přístupu nad jazykem, který
objektově orientované programování nenabízí. Systém GObject nabízí dědičnost,
 vlastnosti objektů i reimplementovatelné metody s dědičností. Dědičnost je 
 řešena vhodně navrženými strukturami v kombinaci s pojmenováváním obslužných 
 funkcí podle dohodnutých pravidel. Přes to, že implemetovaný systém nedosahuje 
 možností skutečně objektového jazyka, jakým je C++, nabízí dostatečné 

Přenos živého záznamu vyžaduje minimalizaci odezvy. 
