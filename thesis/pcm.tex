\chapter{Přenos a formát multimédií}

Před přenosem signálu po internetu nejdříve potřebujeme data zaznamenat a
digitalizovat. Dále je třeba použít takové uložení dat, kterému příjímající
strana dokáže porozumět.

\section{Přenos a kódování hlasu}

V případě audio signálu jako záznamové zařízení obvykle
slouží snímací zařízení, například mikrofon, připojené k převodníku
analogové úrovně napětí na sekvenci diskrétních čísel. Pomocí analogově--digitálního
převodníku získáváme sekvenci čísel s pevnou desetinnou čárkou 
o předem nastavené vzorkovací frekvenci a bitové šířce.
Takové kódování analogového signálu nazýváme pulzně kódová modulace, 
anglicky pulse--code modulation (PCM), pro záznamová zařízení je pravděpodobně
nepoužívanějším kódováním.
Digitální kódování umožňuje ideální
rekonstrukci zaznamenaného signálu po celé přenosové cestě, takže nedochází
k poškození signálu při přenosu. Jistou nevýhodou je omezení zaznamenané 
frekvence na nejvýše polovinu frekvence vzorkovací, pokud nemá být výsledný
signál poškozen.

Nevýhodou PCM je vysoký nutný tok vzhledem k množství přenesené informace\cite{digital_speech}, oproti původnímu analogovému signálu potřebujeme vyšší přenosovou kapacitu. Proto volíme jiné způsoby kódování k ukládání a přenosu řeči.
Kódování dělíme na ztrátové, při jehož použití ztrácíme část přenášené 
informace, a bezeztrátové, kde použijeme výhodnější způsob uložení a odstraníme
nadbytečnou redundanci přenášené informace. Při záznamu hlasu obvykle 
nevyžadujeme bezchybný přenos beze ztráty informace, ale raději omezíme 
množství přenášené informace, abychom mohli komunikovat i~po pomalejší lince,
nebo abychom po ní mohli přenést více toků v přijatelné kvalitě.
Protože ztrátová komprese je zaměřená na vynechání nebo redukci informace, 
kterou lidské
ucho nevnímá, nebo pro pochopení řeči není zásadní, dosahuje ztrátová komprese
výrazně lepší účinnosti, než bezeztrátová. 

Ztrátová komprese hlasu má dva základní přístupy, jakými lze přenášet hlas.
\begin{itemize}
\item{Kódování signálu tak, aby zakódovaný signál měl co nejmenší
odchylku od původního signálu byla co nejmenší. Zakódovaný signál popisuje
tvar vlnění signálu. Jeho výhodou je schopnost popsat kvalitně i obecné zvuky,
například hudbu. Nevýhodou je vyšší bitový tok, tedy nižší účinnost komprese.}
\item{Parametrické kódování signálu, popisující parametry a tvar řečového 
ústrojí člověka, pokud takový zvuk chce vydat. Nazýváme též vokodéry. 
Výhodou je nižší objem 
přenášených informací, nevýhodou obvykle vyšší výpočetní náročnost a nižší 
kvalita.}
\item{Kombinace obou předchozích způsobů.}
\end{itemize}

Protože oba přístupy mají své přednosti, pro vyšší kvalitu přenosu hlasu 
zpravidla oba přístupy kompinujeme v hybridních kodecích. Hybridní kodek
popisuje některé části řeči parametricky, část obtížně popsatelnou několika
parametry (buzení) zakóduje popisem vlnění. V současnosti používané kodeky jsou 
zpravidla hybridní, pokud není potřeba přenášet i neřečové signály,
nebo zachovat kompatibilitu se sítí nepodporující jiné kódování. Například 
pro přenos komunikace analogového faxu je nutné použít kodek určený nejen pro přenos 
lidského hlasu.

Současné moderní algoritmy pro zvýšení účinnosti komprese používají zpravidla
nějakou formu prediktoru, kterým se na základě předešlých dat pokoušejí
popsat změnu oproti předchozímu stavu. Využívají skutečnosti, že především
u znělých souhlásek je signál periodický a~pravděpodobně se skokově nezmění.
Další technikou je vytvoření slovníku fonémů, kde parametricky popíšeme
běžně se vyskytující úseky řeči a při kódování hledáme, kterému vzorku ve 
slovníku je vstupní signál nejvíce podobný. Poté se přenáší pouze odkaz
do slovníku, přijímající strana signál zrekonstruuje na základě vlastního
slovníku.

Algoritmů pro kódování hlasu je řada a stále probíhá vývoj lepších, 
efektivnějších postupů pro získání nižší potřebné přenosové rychlosti 
či menší výpočetní náročnosti, nebo jejich kombinací. 
Protože vývoj nových algoritmů je poměrně
náročný, chrání si vývojáři svoje postupy často patentovou ochranou ve státech,
kde jim to legislativa umožňuje. Existence řady různých algoritmů pro kódování
hlasu není pouze problém čistě technický, v praktických aplikacích hraje roli
také licencování existujících řešení. 

\section{Formát a kódování videa}

Pohybujícím se videem obvykle chápeme sekvenci obrázků zobrazovanou rychle po
sobě. Na rozdíl od zvuku je video vícerozměrná matice jednotlivých bodů obrazu a obvykle potřebuje výrazně vyšší přenosovou kapacitu, než zvukový kanál. Jednotlivé snímky nesou informaci pro každý pixel obrazu, kde pixelem
rozumíme nejmenší viditelný bod v obrazu. 

Pro snímky v odstínech šedé jde o sekvenci úrovní jednotlivých pixelů v definované bitové hloubce, barevné
snímky mají nejčastěji pro každý pixel uvedenou hodnotu jasu a k~ní dvě další
složky definující barvu. Způsobů uložení pixelů v jednom snímku je celá řada,
liší se podle typu kamery, kterou byl snímek pořízen a zobrazovacího zařízení, na který mají být snímky zobrazovány. 

Pro statické obrázky je nejčastější formou uložení pixelů trojice úrovní pro barvy červenou, zelenou a modrou -- RGB. Nejčastěji se setkáme s formátem
osmi bitů pro každou složku, celkem jeden bod má 24 bitů. V některých situacích je výhodné mít ještě další kanál, průhlednost, na jeden pixel použijeme 32 bitů. 

Lidské oko ale nemá stejnou citlivost na všechny barevné složky. Oko je citlivější na intenzitu světla, protože počet tyčinek v oku je výrazně vyšší,
než počet čípků rozlišujících jednotlivé odstíny barev. Při ukládání videa
máme velký počet snímků o velkém počtu snímků, je proto vhodné rozdíly v citlivosti na jednotlivé složky zohlednit. Protože se snažíme dosáhnout co nejmenšího objemu jednoho snímku, pro každý pixel videa častěji používáme
barevný model úrovně jasu a dvou odchylek tónu barvy -- YUV. Hodnotu luminance obvykle ukládáme s vyšším počtem bitů, než složky s barevnou odchylkou, zpravidla mají barevné složky dohromady stejnou velikost, jako
složka jasu.

Kromě velikosti jednotlivých složek se může měnit i způsob uložení v paměti.
První způsob je prokládání jednotlivých barevných složek pixelů za sebou, 
druhou variantou je uložení jedné ze složek všech pixelů za sebou, následující sekvencí pixelů další složky. Protože kombinací uložení je celá
řada, pro popis formátu snímku zpravidla používáme 32 bitové označení složené ze čtyř písmen, Fourcc\cite{fourcc}. Kromě formátu uložení jednotlivých složek rozlišujeme také barevný model obrazu, tedy přiřazení
konkrétních barev určitým úrovním jasu. Mezi jednotlivými barevnými modely můžeme na jiné snadno převádět pomocí matic, s konstantami závisejícími na původu pixelů.

Protože video musí zobrazit několik snímků za vteřinu, bylo by nevýhodné muset pro každý jeden snímek přepočítat z formátu YUV na barevný model RGB,
ve kterém nejčastěji ukládáme barevný obraz na grafické kartě. Aby nebylo nutné přepočítat vždy každý snímek, moderní grafické karty obsahují akceleraci ve formě převodu barevného formátu přímo v kartě. Pro zobrazení snímku potom stačí přenést obrázek ve formátu YUV, a karta zařídí konverzi 
nutnou k zobrazení, obvykle včetně hardwarového škálování obrázku. V grafickém prostředí X11 akceleraci poskytuje rozšíření Xv, příkazem {\em xvinfo} lze vypsat podporované formáty snímků, které lze zobrazovat přímo na kartu.

Aby se zmenšil objem videa, přenášené snímky můžeme komprimovat na úrovni
jednotlivých snímků, aby jeden snímek měl minimální velikost. Příkladem je
kódování JPEG, v případě videa Motion JPEG, kde video je sekvencí nezávislých snímků. Protože v běžném videu se zpravidla nemění scéna skokově, ale postupně, existuje redundance informace mezi jednotlivými snímky. Pro vyšší efektivitu se obvykle používá kromě kódování jednotlivých snímků také kódování mezi snímky, kde přenášíme pouze rozdíl proti několika
předcházejícím snímkům. Obvykle se použije seznam transformací upravujících 
předchozí snímek do nové podoby, kde z původního obrazu použijeme makrobloky
a definujeme jejich posun, rotaci či změnu měřítka. I v případě videa se 
obvykle setkáváme s prediktory, které charakterizují předešlé změny tak,
aby nebylo potřeba přenést všechny transformace, ale pouze změny v pohybu.
Mezisnímkové kódování má nepříjemnou vlastnost, při ztrátě či poškození jednoho snímku
přestanou následné transformace dávat správné výsledky. Proto se přenáší pravidelně klíčový snímek, který nese kompletní snímek, po něm následuje sekvence částečných snímků, které nesou změny proti tomuto snímku.

\section{Přenos v paketové síti}
Pokud chceme přenášet multimédia v paketové síti, je potřeba data rozdělit
na časově ohraničené rámce o velikosti, jež je síť schopná přenést. Protože
pro živý přenos je důležité zachovat minimální zpoždění, používá se zpravidla
pro přenos datagramů bez zajištění bezchybného přenosu.

Data je potřeba přenášet tak, aby obě strany rozuměly přenášeným datům a byly
je schopny správně přehrávat. Prvním předpokladem je použití stejného 
protokolu pro přenos dat a jím definovaný formát přenášených dat. Druhým předpokladem je dohodnutí parametrů přenosu před zahájením vlastního datového
přenosu.

Dohodnutí parametrů přenosu se obvykle provádí jiným spojením, než samotný
datový přenos. K ustavení spojení a dohodnutí parametrů slouží signalizační
kanál. Může být spojen jak protokolem TCP jako v případě XMPP a Jingle,
nebo UDP obvykle používané protokolem SIP, oba už byly popsány v předchozích kapitolách. Pro signalizační kanál je důležité, aby byl schopen fungovat 
na různých konfiguracích sítě, i na sítích používajících překlad adres. Právě
pomocí signalizačního kanálu je umožněno dohodnutí parametrů mezi účastníky
spojení, aby mohlo vůbec započít sestavování přímého spojení. Signalizační
kanál umožní výměnu adres, případně může pomoci při zjišťování topologie
 sítě. Protože signalizace nepřenáší multimediální data, není potřeba jej 
 synchronizovat a dosahovat nízké odezvy, důležitějším parametrem je spolehlivost. Proto není na závadu případné zpoždění TCP/IP spojení při výpadku několika paketů, nebo kaskáda proxy serverů v případě SIP.
 
Datový přenos se navazuje pokud možno přímo mezi koncovými uživateli, 
v případě nemožnosti přímého spojení přes proxy se zvýšením doby odezvy.
Kromě minimální potřebné rychlosti vyžaduje kvalitní přenos malé
 kolísání doby odezvy, tedy jitter. Protože při průchodu internetem obvykle
 nejsme schopni zaručit prioritní zpracování multimediálních dat, kolísání
 nemůžeme eliminovat, ale je možné potlačit jeho vliv, pokud není příliš velké. Kolísání odezvy odfiltruje přidání vstupní fronty, která čeká na přijímaná data po určitou krátkou dobu a vyrovnává tak nárazově přicházející
 data. Nevýhodou je další umělé zvýšení zpoždění přehrávanách dat. Čím větší
je kolísání doby přenosu, tím větší je nutné mít frontu, aby se kompenzoval vliv kolísání. Stále však chceme co nejnižší zpoždění, proto zpoždění 
přenosu i jeho kolísání chceme mít minimální.

Minimálně u zvuku požadujeme navazující datový tok bez mezer\cite{RtpBook}. 
Výpadky mohou nastat z více důvodů. Nejčastějšími důvody je ztracení posílaných dat a příliš velké zpoždění, kdy už fragment nemá smysl přehrát a je zahozen. V IP sítích jsou datové vyšší protokoly, TCP i UDP, chráněny 
kontrolním součtem, který zajišťuje, že nedošlo k poškození přenášeného obsahu. Pokud kontrolní součet nesouhlasí, systém paket zahodí a k vlastní
aplikaci se nedostane ani poškozený paket s částečnou informací. Kvůli 
1 chybnému bitu tak můžeme ztratit celý fragment obsahující 20\,ms řeči.

Protože použitý protokol UDP nezajištuje bezchybný přenos, je nutné nějak
reagovat na ztracené pakety. Aplikace by měla zajistit zotavení z chyb tak,
aby pro koncového uživatele nebyla chyba příliš viditelná. Pokud nepoužijeme 
na straně vysílání prokládání s~přidanými informacemi pro zotavení, nezbývá,
než nahradit chybějící fragment nějakým odhadem chybějícího obsahu. 

V případě výpadku audio signálu existuje řada postupů, obvykle vyžadují 
znalosti vnitřního stavu dekodéru, aby výsledná korekce zněla věrohodně.
Pro znělé hlásky je vhodné nahradit chybějící fragment předchozím fragmentem,
neznělé souhlásky lépe nahradí šumová funkce\cite{RtpBook}. Zotavení z chyb je nedílnou součástí návrh algoritmu a způsobu jeho zabalení v RTP paketu.
Protože komprese zpravidla využívá predikce pro snížení objemu přenášených
zpráv, výpadky části provozu mohou negativně ovlivnit celou řadu správně
doručených datagramů. Odolnost proti chybám v přenosu je dalším parametrem
měřítkem kvality implementace daného algoritmu.
